\setcounter{chapter}{7}
\chapter{Conclusion}\label{Chap:conclusion}

Efforts has been made to implement the halogen chemistry responsible for the observed \acrfull{ode} in the Arctic in the Oslo CTM3. The aim of this thesis was to implement the halogen chemistry, verify the new scheme against observations and to estimate what impact the new scheme had on the \acrfull{rf} globally and over the Arctic.

\medskip

Some of the features of halogen induced \acrshort{ode} can be seen in the results from the BE-branch. The modelled \chem{HBr} and $\chem{O_3}$ are anti-correlated in the 2001 run. The heterogeneous aerosol reaction appears to be the most dominant reaction partaking in the autocatalytic ozone depletion reaction as reactive halogens can be found aloft, but only to a lesser extent at the ground. As some of these components can be seen in the results, this motivates further development of the halogen chemistry causing tropospheric \acrshort{ode}. Furthermore, to obtain a more realistic scheme the heterogeneous chemistry involving chlorine should be included. 

\medskip

The final \acrfull{be}-branch which includes the halogen chemistry necessary to initiate \acrshort{ode} in the Arctic produces highly unstable and varying results regarding the ozone content of the troposphere when compared to observations both at the ground level and in the vertical. The BE-branch shows no significant correlation with ground-based measurements of ozone. Compared to the Original CTM3, the BE-branch produces between 20-30 ppb less $\chem{O_3}$ in the Arctic throughout the months February to June in the 2001-run. Vertical distributions of the ozone mixing ratio (from the 2013-run) shows that the BE-branch is highly variable and sometimes deplete virtually all ozone in the troposphere. Thus, the ozone depletion works, but the scheme is too efficient, leading to too much depletion of $\chem{O_3}$.  


\medskip

Due to diverging results in the 2001 and 2013 run, it is not possible to conclude a specific dependence with altitude regarding tropospheric ozone-induced RF. The temporally (February to June) and globally averaged RF due to tropospheric ozone yielded by the BE-branch demonstrates large deviations between the 2001-run, RF $=-0.012\pm0.12$ Wm$^{-2}$, and the 2013-run, RF $= 0.45\pm0.42$ Wm$^{-2}$. Due to the inconsistency in RF and the fact that the present-day and pre-industrial setup of the BE-branch, these estimates are not correct. Future work is advised to aspire increased stability of the new scheme to avoid the inconsistency seen between the two model runs.  