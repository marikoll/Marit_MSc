\setcounter{chapter}{7}
\chapter{Conclusions}\label{Chap:conclusion}

The final \acrlong{be}-branch which includes the halogen chemistry necessary to initiate \acrlong{ode}s in the Arctic produces highly unstable and varying results regarding the ozone content of the troposphere when compared to observations both at the ground level and in the vertical. The BE-branch shows no significant correlation with ground-based measurements of ozone. 

\medskip

Some of the features of halogen induced \acrshort{ode} can be seen from the BE-branch. The modelled \chem{HBr} and $\chem{O_3}$ is anti-correlated. 

\medskip

The BE-branch produces between 20-30 ppb less $\chem{O_3}$ than the Original CTM3 in the Arctic throughout the months February to June, 2001. 

\medskip

The temporally (February to June) and globally averaged RF due to tropospheric ozone yielded by the BE-branch demonstrates large deviations between the 2001-run, RF $=-0.11\pm0.11$ Wm$^{-2}$, and the 2013-run, RF $= 0.39\pm0.31$ Wm$^{-2}$. Due to the inconsistency in RF and the fact that the present-day and pre-industrial setup of the BE-branch, these estimates are not correct. 