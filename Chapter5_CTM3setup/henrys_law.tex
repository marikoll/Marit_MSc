\begin{table}[ht]
\centering
\begin{tabular}{|lllll|}
\hline
\textbf{Component}          & \textbf{H$^{cp}$ [mol m$^{-3}$Pa]} & \textbf{$\frac{d \ln H^{cp}}{d(1/T)}$ [K]} & \textbf{Note} & \textbf{Reference}                \\ \hline
\chem{HCl} & $1.1\times10^{-2}$                     & $2300$                                         & (*)          & \cite{MARSH1985} \\
\chem{HBr} & $2.4\times10^{-1}$                     & $370$                                          & (**)         & \cite{dean1999}  \\
$\chem{ClONO_2}$            &  $2.1\times10^{5}$.   & $8700$                              & (***)       & \cite{Lelieveld1991TheRO}      \\ \hline
\end{tabular}
\caption{The Henry's law constants are taken from \cite{Sander2015} and references therein. 
\\
(*) Thermodynamical calculation  
\\ 
(**) Only the tabulated data between T = 273 K and T = 303 K from Dean (1992) were used to derive H and its temperature dependence. Above T = 303 K, the tabulated data could not be parameterized very well. The partial pressure of water vapor (needed to convert some Henry's law constants) was calculated using the formula given by Sander et al. (1995). The quantities A and $\alpha$ from Dean (1992) were assumed to be identical. 
\\ 
(***) Assumed to have the same Henry's law as $\chem{HNO_3}$ \cite{TerjePersonal}
\\
\textbf{Note: the units of the Henry's law constants were changed after this implementation (see the Results Section \ref{sec:res_step3}. It was changed to atm M$^{-1}$}}
\label{tab:Henrys_law}
\end{table}