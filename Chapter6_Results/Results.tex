\setcounter{chapter}{5}
\chapter{Results}\label{chap:results}

Tentative layout: 


\section{Code Development}\label{sec:results_code_development}

This section contains some results 

Test-runs are between 14- and 20 (model) days, production runs are 1-3 (model) months

\subsection{A First Look}

Figure \ref{fig:CompObsOrigBE} shows results in terms of $\chem{O_3}$-concentration from preliminary model runs with the chemistry described in Chapter \ref{Chap:CTM3theory_ocean_hetReact} and the branches listed in Section \ref{sec:code_availability}. 

\medskip

To verify the results, I have used the measurements of $\chem{O_3}$ and \chem{HBr} available (see Appendix \ref{app:ebas_noaa_data}), as well as \chem{BrO} measurements from literature for comparison. Ozone is used as a proxy as it was reproducability of the ODEs seen in measurements I was interested in. The \chem{HBr} measurements should in theory correspond to elevated concentrations after an ODE according to the box-model results by \cite{CAO}. Finally, \chem{BrO}-concentrations should be anti-correlated with the depletion of ozone (\cite{barrie}). 

\medskip

Branch \ref{def:BE_PD} produces very low concentrations of $\chem{O_3}$, as can be seen from Figure \ref{fig:CompObsOrigBE}. It does not capture the ozone depletion events that can be seen for instance at Alert around the 9th of April. The original CTM3 branch produced $\chem{O_3}$ concentrations more comparable to observations, although without distinct bromine explosion events. 

\begin{figure}
    \centering
    \includegraphics[width = \linewidth]{Chapter6_Results/images/ozone_2001_compObsOrigBE.png}
    \caption{Ozone measurements (black line) and model results from the original CTM3 (Branch \ref{def:origCTM3_PD}) (blue line) and Branch \ref{def:BE_PD} (turquoise line) at the four different stations, Alert (top left), Barrow (top right), Summit (lower left) and Zeppelin (lower right) with available measurements in 2001. Model results were taken from the approximate altitude of the station in hPa. PD = present day, BE = bromine explosion}
    \label{fig:CompObsOrigBE}
\end{figure}


\subsection{Test: Removing Heterogeneous Reactions}

Figure \ref{fig:test_RemoveHetReacts} shows results in terms of $\chem{O_3}$-concentration from attempting to turn off different heterogeneous reactions, namely snow/ice reactions (described in Section \ref{sec:snow_ice_react}), heterogeneous reactions over aerosol surfaces (described in Section \ref{sec:aerosol_react}) as well as heterogeneous reactions involving chlorine and bromine, respectively. The runs were initiated with the same restart file (spin-up) as Branch \ref{def:BE_PD}. For this purpose, four new branches were created (for a full overview of the branches, see Table \ref{tab:branches}). These were:

\begin{mydef}\label{def:BE_PD_noAerosol}
    \texttt{marikoll\_bromine\_explosion\_noHetAerosol}: Branch \ref{def:BE_PD} without heterogeneous aerosol reactions (Reaction \ref{R:8}). 
\end{mydef}

\begin{mydef}\label{def:BE_PD_noIce}
    \texttt{marikoll\_bromine\_explosion\_noSnowIce}: Branch \ref{def:BE_PD} without heterogeneous reactions over ice surfaces (called \texttt{noSnowIce} by mistake) (Reaction \ref{R:7}).
\end{mydef}

\begin{mydef}\label{def:BE_PD_noCl}
    \texttt{marikoll\_bromine\_explosion\_noHetChlorine}: Branch \ref{def:BE_PD} without heterogeneous reactions involving chlorine (Reaction \ref{R:8} and \ref{R:7} with $\chem{X} = \chem{Cl}$).
\end{mydef}

\begin{mydef}\label{def:BE_PD_noBr}
    \texttt{marikoll\_bromine\_explosion\_noHetBromine}: Branch \ref{def:BE_PD} without heterogeneous reactions involving bromine (Reaction \ref{R:8} and \ref{R:7} with $\chem{X} = \chem{Br}$).
\end{mydef}


\begin{table}
\centering
\begin{tabular}{|ll|}
\hline
\textbf{Branch}                                      & \textbf{Reference}          \\ \hline
\texttt{marikoll\_originalCTM3\_NoStrat}             & \ref{def:origCTM3_PD}     \\
\texttt{marikoll\_originalCTM3\_noStrat\_pi}         & \ref{def:origCTM3_PI}     \\
\texttt{marikoll\_bromine\_explosion\_susanne}       & \ref{def:BE_PD}           \\
\texttt{marikoll\_bromine\_explosion\_PI}            & \ref{def:BE_PI}           \\
\texttt{marikoll\_bromine\_explosion\_noHetAerosol}  & \ref{def:BE_PD_noAerosol} \\
\texttt{marikoll\_bromine\_explosion\_noSnowIce}     & \ref{def:BE_PD_noIce}     \\
\texttt{marikoll\_bromine\_explosion\_noHetChlorine} & \ref{def:BE_PD_noCl}      \\
\texttt{marikoll\_bromine\_explosion\_noHetBromine}  & \ref{def:BE_PD_noBr}      \\ \hline
\end{tabular}
\caption{Overwiew of brances used in the developing process. References refer to chapter and branch number}
\label{tab:branches}
\end{table}


\begin{figure}
    \centering
    \includegraphics[width = \linewidth]{Chapter6_Results/images/ozone_2001_testRemoveHetReacts.png}
    \caption{Caption}
    \label{fig:test_RemoveHetReacts}
\end{figure}


\medskip

In the vertical, the corresponding $\chem{Br_y}$ concentrations to Figure \ref{fig:test_RemoveHetReacts} are shown in Figures \ref{fig:vert_noAer_bry_2001}-\ref{fig:vert_noBr_bry_2001} for the four new branches, Branch \ref{def:BE_PD_noAerosol}-\ref{def:BE_PD_noBr}, respectively (The $\chem{Br_y}$-family is explained in Section \ref{sec:halogen_families_BryClxCly}). 

\begin{figure}
    \centering
    \includegraphics[width = \linewidth]{Chapter6_Results/images/noAerosol_2001_bry.png}
    \caption{Modelled $\chem{Br_y}$ without the heterogeneous aerosol reactions. The y-axis shows altitude up to 600 hPa at each station with ozone measurements at 12:00 (UTC) (Alert, Barrow, Zeppelin and Summit) in 2001.}
    \label{fig:vert_noAer_bry_2001}
\end{figure}

\begin{figure}
    \centering
    \includegraphics[width = \linewidth]{Chapter6_Results/images/noSnowIce_2001_bry.png}
    \caption{Modelled $\chem{Br_y}$ without theheterogeneous reactions over ice surfaces. The y-axis shows altitude up to 600 hPa at each station with ozone measurements at 12:00 (UTC) (Alert, Barrow, Zeppelin and Summit) in 2001.}
    \label{fig:vert_noSnowIce_bry_2001}
\end{figure}

\begin{figure}
    \centering
    \includegraphics[width = \linewidth]{Chapter6_Results/images/noCl_2001_bry.png}
    \caption{Modelled $\chem{Br_y}$ without the heterogeneous reactions involving chlorine. The y-axis shows altitude up to 600 hPa at each station with ozone measurements (Alert, Barrow, Zeppelin and Summit) in 2001.}
    \label{fig:vert_noCl_bry_2001}
\end{figure}

\begin{figure}
    \centering
    \includegraphics[width = \linewidth]{Chapter6_Results/images/noBr_2001_bry.png}
    \caption{Modelled $\chem{Br_y}$ without the heterogeneous reactions involving bromine. The y-axis shows altitude up to 600 hPa at each station with ozone measurements at 12:00 (UTC) (Alert, Barrow, Zeppelin and Summit) in 2001.}
    \label{fig:vert_noBr_bry_2001}
\end{figure}




\textbf{NOTE:} after these tests, a mistake in the scaling of $\chem{Cl_x}$ was discovered. \chem{BrCl} had mistakenly been scaled with this family, which led to disappearance of all \chem{Cl}-species. The subsequent tests were fixed for this.

\subsection{Integrating $\chem{Cl_y}$ in Branch \ref{def:BE_PD}}

Figure \ref{fig:test_ClyInt} contains the result in terms of $\chem{O_3}$ concentration where an integration of the $\chem{Cl_y}$-family was added to the initial BE-branch (Branch \ref{def:BE_PD}) This integration was not handled prior to the previous model runs (the chemical families are listed in Section \ref{sec:halogen_families_BryClxCly}). 

\medskip

As the inclusion of $\chem{ClONO2}$ (Reaction \ref{R:clono2}) function purely as a sink to the $\chem{ClO}$, this reaction was also removed altogether to see if this would help the low chlorine levels. 

\begin{figure}
    \centering
    \includegraphics[width = \linewidth]{Chapter6_Results/images/ozone_2001_newClyIntegration.png}
    \caption{Ozone measurements (black line) and model results from the original CTM3 (orange line), Branch \ref{def:BE_PD} (blue line) and the attempt to integrate the $\chem{Cl_y}$ family (green line) at the four different stations, Alert (top left), Barrow (top right), Summit (lower left) and Zeppelin (lower right) with available measurements in 2001. Model results are taken from the first model level at $998.82 hPa$. PD = present day, BE = bromine explosion}
    \label{fig:test_ClyInt}
\end{figure}

\subsection{Development of Branch \ref{def:BE_PD_noCl} Without Heterogeneous Chlorine Reactions}



\subsubsection{Changing $L_{mix}$ in Reaction \ref{R:7}}

The boundary layer height, $L_{mix}$ and deposition velocity, $v_d$ in the preliminary runs had values listed in Section \ref{sec:impl_multiphase_react}. In an attempt to "kick-start" the bromine explosions $L_{mix}$ was lowered (And corresponding $v_d$ increased) in the \texttt{marikoll\_bromine\_explosion\_noHetChlorine}- branch. 

\medskip

The first test was executed with $L_{mix} = 25 m$ and $v_d = 0.00824 m/s$. This boundary layer height was chosen due to the height of the second model layer height $995.86 hPa$, with is about $23 m$ higher than the first model layer height at $998.82 hPa$. The low level was chosen to make sure the bromine explosion would indeed occur. This test caused too much bromine explosion for the model to complete the run. $L_{mix}$ was then lowered to $L_{mix} = 100 m$ with $v_d = 0.00667 m/s$. 

\subsubsection{Lmix = 200}

\subsubsection{Lmix = 100}

\subsubsection{Lmix = 100 After Fix}

The weighing of Reaction \ref{R:7} was removed (in \texttt{pchemc\_ij.f90}). It was initially weighted with $0.5$ assuming half of the \chem{HOBr} was reacting with \chem{Br} and half with \chem{Cl}, but was removed as the heterogeneous reactions involving \chem{Cl} were removed. Also, $\chem{Br_2}$ was added to the debugging-scaling in \texttt{pchemc\_ij.f90}

\subsubsection{Lmix = 100, \chem{HBr} Adjusted to 30 ppt}

In order to boost the concentration of \chem{HBr}, the concentration was hard-coded to 30 ppt ($= 8.059 \text{molecules}cm^{-3}$ at $273.15 K$) in the first sub-timestep of \texttt{pchemc\_ij.f90}. This worked in the test-run


\subsubsection{Lmix = 200, \chem{HBr} Adjusted to 10 ppt}

A test-run of this managed to maintain 10 ppt - concentrations of \chem{HBr}, but longer runs crashed. Because the 10 ppt was maintained, a restart file was kept from the test-run to boost the concentration in a longer run (without hard-coding of the \chem{HBr} concentration)


\subsubsection{Lmix = 100, with new restart file}

C3RUN\_BE\_HFOUR\_HBr10ppt\_May

Figur med HOBr og HBr - antikorrelert. Null effekt på ozon, hot spots med Br, BrO og Br2. 

The anticorrelation of \chem{HOBr} and \chem{HBr} indicates that \chem{HOBr} is titrated from the system, leaving hot spots of \chem{HBr}. 

\subsubsection{Lmix = 50, with new restart file}

Still anti-correlation between \chem{HOBr} and \chem{HBr}. Maybe a bit less $\chem{O_3}$? 

vd = 0.0074 

C3RUN\_BE\_HFOUR\_HBr10ppt\_Lmix50\_May

\subsubsection{Lmix = 25, with new restart file}

vd = 0.00824

C3RUN\_BE\_HFOUR\_HBr10ppt\_Lmix25\_May


\subsubsection{Lmix = 100, cycling of \chem{HOBr} and \chem{HBr} and hard-coded photodissociation rates}

C3RUN\_BE\_HFOUR\_HBr10ppt\_Lmix100\_ohbr2 - test run with the reaction: 

\begin{reaction}
    \chem{Br_2} + \chem{OH} \rightarrow \chem{HOBr} + \chem{Br}
    \label{rqn:oh_br2}
\end{reaction}

C3RUN\_BE\_HFOUR\_HDP\_MarchMay - long run with the reaction: 

\begin{reaction}
    \chem{HBr} + \chem{OH} \rightarrow \chem{H_2O} + \chem{Br}
    \label{rqn:oh_hbr}
\end{reaction}

As well as hard-coded photodissociation rates as it turned out the rates were not previously calculated for Reactions \ref{R:18}, \ref{R:20} and \ref{R:12}. These were previously set to be solved by the fast-JX method (see Section \ref{sec:CTM3_photochemistry}), but this had not worked. Instead, these were hard-coded as was done previously (by \cite{Susanne}) for Reactions \ref{R:19} and \ref{R:1}. The photodissociation rates were set to: 

\begin{itemize}
    \item $3\cdot10^{-4} s^{-1}$ for Reaction \ref{R:18} (value from \cite{CAO})
    \item $0.014 s^{-1}$ for Reaction \ref{R:20}(value from \cite{CAO})
    \item $0.05\cdot10^{-8} s^{-1}$ for Reaction \ref{R:12} (value from \cite{Papanastasiou2013}, Arctic spring dissociation rate, Figure 2, p. 3022)
\end{itemize}


Testen viser bedring, men \chem{HBr} er fortsatt altfor høy- for treig våtavsetning? 

\subsubsection{Lmix = 100, new Henry's law constants for \chem{HBr} and \chem{HCl}}\label{sec:new_henrys_law}

New values for Henry's law constants were applied to fix the high concentrations seen in \chem{HBr}. The values were taken from \cite{dean1999}:

\begin{itemize}
    \item \chem{HBr}: $2.5 \cdot 10^{1} [M/amt]$, $370 K$
    \item \chem{HCl}: $1.9\cdot10^1 [M/atm]$, $600 K$ 
\end{itemize}

Fin ODE klokka 21 den 29/3! Høye Br og BrO og lav ozon! (fil 88) 


\subsubsection{Test at HTWO}

This was with both hard-coded photodissosiation rates and higher Henry coefficient C3RUN\_BE\_HTWO\_HX\_HDP\_MarchApril

\section{Comparison between the PD- and PI-branches}

\section{Comparison with station data}

\section{Comparison with literature}

\cite{Zhao2016} measured BrO column at Eureka in 2011 -- SJekk ut! 


\cite{Peterson2016} and \cite{Peterson2015} -- BrO over Barrow -- sjekk ut! 

Measurements of $\chem{Br_2}$, \chem{BrCl} and $\chem{O_3}$ were conducted by \cite{Foster2001} at Alert research station. They found $\chem{Br_2}$ mixing ratios up to $\sim$ 25 \acrshort{ppt} and \chem{BrCl} at mixing ratios up to 35 \acrshort{ppt} between day 40 and 75 in 2001. Ozone was depleted from background values of $\sim$ 30-40 \acrshort{ppb} to below 10 ppb. 

\medskip

\cite{Simpson2017} investigated the \chem{BrO} column using \acrlong{maxdoas} instrumentation near Barrow in 2012.

\medskip

\cite{Luo2018} also investigated the \chem{BrO} column using \acrshort{maxdoas} in Ny Ålesund in 2015. 

\medskip

\cite{Thomas2012} and \cite{Thomas2011} about the mechanism behind ODEs at Summit, Greenland. 

\section{Calculation of radiative forcing using PD- and PI model results}