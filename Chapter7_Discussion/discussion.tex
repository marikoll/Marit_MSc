\setcounter{chapter}{6}
\chapter{Discussion}\label{chap:discussion}

\section{Code Development}\label{sec:discussion_code_development}

The Oslo CTM3 is documented by \cite{SovdeManual} and the documentation in the code itself. The branches I developed were based on the work already performed by \cite{Susanne} in her master thesis from 2016. The process was slowed down by quite some time due to the lack of documentation in the new code and my ability to interpret it. Also, due to the change of super computers in January 2020, some problems arose concerning how to optimize the model runs in technical terms (see Section \ref{app:supercomputer}). This led to the decision to perform the model runs at \texttt{HFOUR}-resolution instead of \texttt{HTWO} and shorten the time of the spin-up and the model runs to three months (model time). 



\subsection{A First Look}

In Figure \ref{fig:CompObsOrigBE} (in Section \ref{sec:results_code_development}) the ozone observations at Alert, Barrow, Summit and Zeppelin were compared to model results using the original CTM3 branch (Branch \ref{def:origCTM3_PD}) and the bromine explosion branch (Branch \ref{def:BE_PD}). The latter produced far too low $\chem{O_3}$-concentrations. In order to examine the reason why, a test where the different types of heterogeneous reactions were removed was performed. This is explained in the next section. 

\subsection{Test: Removing Heterogeneous Reactions}

Branch \ref{def:BE_PD_noAerosol}-\ref{def:BE_PD_noBr} were created as an attempt to see which process may have caused the problems in the initial BE-branch (Branch \ref{def:BE_PD}). The resulting modelled ozone along with ozone measurements are shown in Figure \ref{fig:test_RemoveHetReacts}. This had to be seen in the context of the modelled bromine content as there generally was not enough bromine to create such low values of ozone. 


no chlorine branch chosen because it was less invasive to adjust this, even though the branches without aerosol reactions and het. bromine reactions were much closer to the original in terms of ozone. 


\section{Future work}

The acidity of the reaction surface for the heterogeneous reactions are not considered, although it is highly dependent on pH. 

\medskip
Sea salt debromination increase from pre-industrial times to today due to enhanced particle acidity from present day emmissions are not considered here (\cite{Parella})

\medskip
The implementation in Section \ref{sec:oceanic_emissions} does not take into consideration antropogenic emissions of organic halocarbons, seasonality in emissions or the difference in lifetime for $\chem{CHBr_3}$ and $\chem{CH_2Br_2}$. The implementation is a latitudinal fixed parameterization, and cannot thus not interchange with existing inventories in the model. It would have been better to use an NetCDF-based emission field. 
\medskip
In the implementation of the heterogeneous reactions over snow and ice (Section \ref{sec:tropchem_oslo}), the existence of sea ice is the only variable determining the occurrence of the reaction. To improve this, the age of the ice (multiyear ice or newly formed ice), snow cover and ph should be considered (e.g. \cite{Peterson2019}).

\medskip

The model should also be verified for the Antarctic, which also experiences ODEs (\cite{Simpson2015}).


\medskip

The parameterization of the aerodynamic resistance $r_a$ in Sections \ref{sec:snow_ice_react} should have been calculated using values for the boundary layer conditions, wind speed and boundary layer height, that are in the CTM3 already, rather than prescribed values from \cite{CAO}. 