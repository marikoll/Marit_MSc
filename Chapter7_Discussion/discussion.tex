\setcounter{chapter}{6}
\chapter{Discussion}\label{chap:discussion}

This chapter discusses the results seen in Chapter \ref{chap:results}. The chapter starts with a section concerning the development of the implementation of the halogen chemistry responsible for the \acrfull{ode} seen in the Arctic (Section \ref{sec:discussion_code_development}). The resulting implementation is called the BE (bromine explosion) branch, and is compared with observations and the Original CTM3 in Section \ref{sec:disc_final_Version}. Lastly, the ozone induced tropospheric \acrfull{rf} calculated by using the BE-branch and the Original CTM3 is analysed. 


\section{Code Development}\label{sec:discussion_code_development}

The Oslo CTM3 documentation consist of the CTM3 manual(\cite{SovdeManual}) and the inline code documentation. The branches developed in this thesis were based on the work done by \cite{Susanne} in her master thesis from 2016. Interpreting the new code was a challenge. The code was poorly documented, which caused the process to be slowed down with quite some time. Also, due to the change of super computers in January 2020, some problems arose concerning how to optimize the model runs in technical terms (see Section \ref{app:supercomputer}). This led to the decision to perform the model runs at \texttt{HFOUR}-resolution instead of \texttt{HTWO} and shorten the time of the spin-up and the model runs to three months (model time). 

\medskip

This section covers the many of the test-results which led to the finalized version of the halogen branch. For the sake of validating the tests, results are compared against observational data of $\chem{O_3}$, \chem{HBr} and \chem{BrO} taken from the following studies/sites: 

\begin{itemize}
    \item Ozone measurements from EBAS for the stations with available data for 2001 (Alert, Barrow, Summit and Zeppelin) (see Figure \ref{tab:ebas_noaa_data_taken})
    \item Filterable \chem{Br} (\chem{f-Br}) measurements from Alert (see Figure \ref{fig:Barrie_1988} taken from \cite{barrie}). \chem{f-Br} is assumed to be approximately 50-93\% is gaseous \chem{HBr} (\cite{barrie})
    \item \chem{BrO}-\acrfull{vcd} measurements from Barrow (see Figure \ref{fig:Peterson_2015} taken from \cite{Peterson2015})
\end{itemize}

In Figure \ref{fig:CompObsOrigBE}, Section \ref{sec:results_code_development}, the ozone observations at Alert, Barrow, Summit and Zeppelin were compared to the model results using the original CTM3 branch (Branch \ref{def:origCTM3_PD}) and the bromine explosion branch (Branch \ref{def:BE_PD}). The latter produced far too low $\chem{O_3}$-concentrations even though the bromine content was too low to justify the low ozone-concentration (not shown here). In order to examine the reason why, a test where the different types of heterogeneous reactions were removed was performed. This is explained in the next section. 

\subsection{Test: Removing Heterogeneous Reactions}

Branch \ref{def:BE_PD_noAerosol}-\ref{def:BE_PD_noBr} were created as an attempt to see which process may have caused the problems in the initial BE-branch (Branch \ref{def:BE_PD}). The setup behind the tests is explained in Section \ref{subsec:res_remove_het_reacts}. The resulting modelled ozone along with ozone measurements are shown in Figure \ref{fig:test_RemoveHetReacts}. When the heterogeneous reactions over ice surfaces were removed, the resulting modelled ozone content became similar to what was produced originally (by Branch \ref{def:BE_PD}), indicating that the problem had to be elsewhere. The other three branches produced ozone with comparable mixing ratios as what was observed. Branches \ref{def:BE_PD_noAerosol} and \ref{def:BE_PD_noBr} resembles each other concerning the ozone-concentration, whereas Branch \ref{def:BE_PD_noCl} deviates and shows more sign of having some ozone-depletion patterns (e.g. around the 9th of April at Zeppelin). 

\medskip

The branch without heterogeneous chlorine reactions (Branch \ref{def:BE_PD_noCl}) was chosen for further development as this was considered less invasive to the halogen-chemistry than to exclude the heterogeneous bromine chemistry or the heterogeneous aerosol chemistry. 

\medskip

Figures \ref{fig:vertHBr_noCl}-\ref{fig:polarHBr_noCl} shows the \acrfull{vmr} in with altitude and the concentration in the Arctic at the first model layer, respectively. Compared to the measurements of filterable bromine, \chem{f-Br}, performed by \cite{barrie}, the concentration shown in Figure \ref{fig:polarHBr_noCl} is too low (see Figure \ref{fig:Barrie_1988} in the Appendix). The same applies to the \chem{BrO} \acrshort{vcd} over the lowermost $\sim 250$ m shown in Figure \ref{fig:polarBrO_noCl}. Compared to the \acrshort{maxdoas}-measurements performed by \cite{Peterson2015} (see Figure \ref{fig:Peterson_2015} in Appendix) these are about a factor of $10^7$ too low. These low concentrations were focused upon in the further development of the branch. 
%This had to be seen in the context of the modelled bromine content as there generally was not enough bromine to create such low values of ozone. 



\subsection{Development of Branch \ref{def:BE_PD_noCl} Without Heterogeneous Chlorine Reactions}


\subsubsection{Initializing Branch \ref{def:BE_PD_noCl} With a Higher \chem{HBr} Concentration}\label{sec:disc_step2}

Figure \ref{fig:ozone_noCl_step2} contains the resulting ozone \acrshort{vmr} at the four different stations with ozone measurements from 2001 after performing the following tests on Branch \ref{def:BE_PD_noCl}: 

\begin{itemize}
    \item Hard-coding the \chem{HBr}-concentration to being constantly 30 ppt 
    \item Hard-coding the \chem{HBr}-concentration to being constantly 10 ppt
    \item Initializing the run with a restart file (spin-up) from the run above (\chem{HBr}-concentration hard-coded to 10 ppt)
\end{itemize}

The first two tests were performed for the purpose of investigating whether a forced high concentration of \chem{HBr} would have any effect on the content of reactive bromine species and therefore possibly the production of \acrshort{ode}. From Figure \ref{fig:ozone_noCl_step2} it is clear that the ozone \acrshort{vmr} is affected, with especially the forced 10 ppt-run obtaining observed $\chem{O_3}$ values to a larger extent than before. 

\medskip

The run initialized with a restart file from the 10 ppt-\chem{HBr} concentration run was performed to allow the system to run in a more physical sense (i.e. avoid hard-coding) although containing more bromine. This run generally produced ozone concentrations that were slightly higher than measured. The \chem{HBr}-\acrshort{vmr} and concentration seen in Figures \ref{fig:vertHBr_newRestart}-\ref{fig:polarHBr_newRestart}, respectively, show that the content is about an order of magnitude to high compared to what was seen in Figure \ref{fig:Barrie_1988}. Thus, the bromine content has been boosted by the new initialization, but without any clear effects on the ozone concentrations. 

\medskip

The \chem{HBr}- and \chem{HOBr}-concentrations are closely linked and anti-correlated, as can be seen from Figures \ref{fig:polarHBr_newRestart}-\ref{fig:polarHOBr_newRestart}. A hypothesis for this behaviour could be that the \chem{HOBr} was titrated from the system, leaving hot spots of \chem{HBr}. The subsequent testing thus included two more reactions (Reactions \ref{rqn:oh_br2} and \ref{rqn:oh_hbr}) in order to cycle these species more efficiently. 

\medskip

The \chem{BrO}-\acrshort{vcd} shown in Figure \ref{fig:polarBro_newRestart} is still about 6 orders of magnitude too low compared to what could be expected from Figure \ref{fig:Peterson_2015}. The areas of elevated \chem{BrO} are, however, related to areas of elevated \chem{HOBr} in Figure \ref{fig:polarHOBr_newRestart}, indicating that Reaction \ref{R:7} is indeed working. The low content could then be related to the inefficient cycling of \chem{HOBr}. 


\subsubsection{Hard-Coding Photodissociation and Adjusting the Henry'Law Coefficient}\label{sec:disc_step3}

Figure \ref{fig:ozone_2001_step3} contains the modelled results of four new tests, as well as the observational data, the original CTM3 and the last test from the previous section initialized with a \chem{HBr}-concentration of 10 ppt. The new tests included: 

\begin{itemize}
    \item Hard-coded photodissociation rates (Hard-coded P) as it turned out the photolysis of the following reactions were in fact not working prior to this\footnote{The rest of the tests contained these hard-coded rates. }
    \begin{itemize}
        \item $3\times10^{-4}$ s$^{-1}$ for Reaction \ref{R:18} (value from \cite{CAO})
        \item $0.014$ s$^{-1}$ for Reaction \ref{R:20}(value from \cite{CAO})
        \item $0.05\times10^{-8}$ s$^{-1}$ for Reaction \ref{R:12} (value from \cite{Papanastasiou2013}, Arctic spring dissociation rate, Figure 2, p. 3022)
    \end{itemize}
    
    \item A new Henry's law coefficient, as it turned out the previous coefficient had the wrong unit. The low Henry coefficient refers to:
    \begin{itemize}
        \item \chem{HBr}: $7.2\cdot 10^{-1} [M/amt]$, $6100 K$ (Taken from: \cite{Chameides1992})
    \end{itemize}   
    \item The high Henry's law coefficient refers to: 
    \begin{itemize}
        \item \chem{HBr}: $2.5 \cdot 10^{1} [M/amt]$, $370 K$ (Taken from: \cite{dean1999})
    \end{itemize}
    \item The higher Henry's law coefficient was finally kept in the final version of the CTM3, and was therefore used in the \texttt{HTWO}-test
\end{itemize}

The resulting modelled ozone \acrshort{vmr} (Figure \ref{fig:ozone_2001_step3}) from these runs is lower than the previous tests, although with variations (as opposed to the first run using Branch \ref{def:BE_PD} in Figure \ref{fig:CompObsOrigBE}). It could be expected that these tests produced lower ozone-concentrations, as especially the photodissociation rates are a key point in the ozone depletion at the point of Arctic spring, causing halogens to become reactive. The change in the Henry's law coefficient was essential as the previous implementation was wrong, but the high- and low coefficient produce quite similar $\chem{O_3}$-\acrshort{vmr}. 

\medskip

The \chem{HBr}-\acrshort{vmr} in the vertical and the Artcic concentration in Figure \ref{fig:vertHBr_HTWO_step3} and \ref{fig:polarHBr_HTWO_step3}, respectively shows that the \chem{HBr} concentration is still an order of magnitude too high compared to the findings by \cite{barrie} in Figure \ref{fig:Barrie_1988}. Furthermore, the anti-correlation between \chem{HBr} and \chem{HOBr} can still be seen in Figures \ref{fig:polarHBr_HTWO_step3} and \ref{fig:polarHOBr_HTWO_step3}. This suggests that \chem{HOBr} is still being titrated from the system. 

\medskip

Figure \ref{fig:polarBrO_HTWO_step3} shows that the \chem{BrO}-\acrshort{vcd} is still about five orders of magnitude lower than what was found by \cite{Peterson2015} (see Figure \ref{fig:Peterson_2015} in the Appendix). Thus, the \acrshort{vcd} has increased compared to the previous tests, but is still not quite the magnitude it should be, compared to litterature. 

\medskip

There are indications of a possible relation between the high \chem{HOBr}-concentrations at Alert and the low ozone \acrshort{vmr} seen at the same time period in Figure \ref{fig:ozone_2001_step3}. It can also be seen from \ref{fig:polarBrO_HTWO_step3} that there is an elevated \acrshort{vcd} of \chem{BrO} at Alert in the same  time-period

\subsection{Higher Henry's Law Coefficient and Higher Photodissociation of \chem{HOBr}}\label{sec:disc_step4}

Originally, the high Henry coefficient from the previous section was intended to be the one used in the final version (see Figure \ref{fig:ozone_2001_step3}). However, in order to perform the final runs in a way that would be reasonable to compare against the original version of the CTM3 as well as use in calculations of RF, the restart file provided from \cite{StefaniePersonal} (see Section \ref{subsec:restart_files}) was to be used for the 6 months runs at \texttt{HTWO} for all versions (\acrfull{pi} with original CTM3 version and \acrfull{be}-version, \acrfull{pd} with original CTM3 version and \acrshort{be}-version). The model crashed when running the final version of the \acrshort{be}-branch, obtaining negative values of \texttt{ISOR1} (a radical species) after 17 hours run time (i.e. not model time). After analyzing the results that the model was able to produce before crashing, I decided to attempt increasing the wet deposition of \chem{HBr} (increasing Henry's law coefficient) as well as increasing the photodissociation of \chem{HOBr}. The new Henry's law constant for the wet deposition of \chem{HBr} was taken from \cite{Sander99}: 

\begin{itemize}
    \item $1.3\times10^9/K_\text{A} [M/atm]$, $10 000 K$ (Taken from: \cite{Brimblecombe1988TheSA})
    \item The acid dissociation constant, $K_\text{A}$, was taken as $\ln{K_\text{A}} \approx 9.8$ (\cite{Levanov})
\end{itemize}

And the new photodissociation rate for \chem{HOBr} became:

\begin{itemize}
    \item $3\times10^{-3}$ s$^{-1}$ for Reaction \ref{R:18} (based on value from \cite{CAO}, but an order of 10 faster)
\end{itemize}

The \acrshort{pi} \acrshort{be}-version had been able to run with the version from the section above, but experienced the same problem when running with the version in this section. Thus, the 6-month run from the previous section was kept. This does not serve as a good basis for calculating a trustworthy RF, but due to time limitations, this was what was used. 

\medskip

The original version of the CTM3 was able produce the model runs as planned, both for \acrshort{pd}- and \acrshort{pi} runs. 

\subsubsection{Analysis of the Final Branch}\label{sec:disc_finalBRanch}


Figure \ref{fig:ozone_2001_step4} shows the resulting ozone mixing at the four different stations. The final version (green line) is the version that will be used for further analysis. The initial lower mixing ratios produced by the final version could be due to the need for spin up, as the restart file initializing the model contains a chemistry that differs from what I have implemented. The ozone mixing ratios in the restart file used are normal background values, but the chemistry implemented seems to be depleting it all initially. January has been left out of the following analysis, as it is assumed to be corrupted by the lack of spin up. The stabilization of the ozone \acrshort{vmr} around 10-20 ppb is a bit low at all stations. It is comparable to what was measured at Alert and Barrow, however with less oscillations. At Summit, the $\chem{O_3}$-\acrshort{vmr} is about 30 ppb too low throughout the time period.


\medskip

In Figure \ref{fig:polarHBr_step4}, the \chem{HBr}-concentration in the first model layer (in gm$^{-3}$) is comparable to what was found by \cite{barrie}. The \chem{BrO}-\acrshort{vcd} in Figure \ref{fig:polarBrO_step4}, however reveals that the \chem{BrO}-\acrshort{vcd} is about $10^5$molecules cm$^{-2}$ too low compared to what was found by \cite{Peterson2015}\footnote{\cite{Peterson2015} found \chem{BrO}-\acrshort{vcd} on the order of $10^{13}$molecules cm$^{-2}$ at Barrow in Marc/April 2012, see Figure \ref{fig:Peterson_2015}}. It seems, however, from Figure \ref{fig:polarHOBr_step4} that \chem{HOBr} is still being titrated from the system by Reaction \ref{R:7} and \ref{R:8}, and not efficiently recycled by Reaction \ref{rqn:oh_br2}. Only patches of concentrations reaching $1.0\times10^{-7}$ gm$^{-3}$. The polar ozone concentration (in Figure \ref{fig:polarO3_step4}) does seem to correspond well with the \chem{HBr}-concentration with lower concentrations towards the Bering Strait on the 8th of May moving eastward, and higher concentrations East of Svalbard moving westward. This also corresponds to the higher \chem{BrO}-vcd seen on the 10th of May (Figure \ref{fig:polarBrO_step4}), which is situated more or less above the patch of low $\chem{O_3}$ concentrations seen towards Barrow and Siberia on the 10th of May.\footnote{Keep in mind that Figures \ref{fig:vertHBr_step4}-\ref{fig:polarBrO_step4} are only snapshots of the state of the atmosphere used to verify the halogen-implementation} 

\section{Analysis of the Final Version of the Halogen Branch}\label{sec:disc_final_Version}

Due to the noise appearing in the final version of the halogen chemistry implemented in the CTM3 (henceforth called the BE-branch) seen in the ozone results in Figure \ref{fig:polarO3_step4}\footnote{Ozone timeseries were also made for 2013 (Figure \ref{fig:ozone_2013}), but as I only had data for one day each seventh day, the figure is not used for analysis. It can be found in Appendix \ref{app:final_version}}, January was assumed to be too affected by the diverging chemistry in the restart-file used to be taken into consideration, and was therefore discarded. Ideally, a restart-file with a longer spin-up should have been used, but unfortunately, I ran out of time. Thus, the resulting data was split up into two periods, Period 1 (February 1st-April 24th) and Period 2 (April 24th-June 30th) as shown in Figure \ref{fig:2p_step5}. This was done because Period 1 and 2 seems to be affected by different regimes with lower ozone \acrshort{vmr} in Period 1 and higher in Period 2.

\subsection{Analysis of the Two Periods February-April and April-June}\label{sec:disc_twoPeriods}

The temporal evolution of the ozone \acrshort{vmr} in Period 1 of 2001 produced by the BE-branch at the ground level \footnote{The model ground level Summit and Zeppelin are assumed to be at their approximate altitude in pressure} (Figure \ref{fig:2p_step5}) shows that the BE-branch produces far too low ozone \acrshort{vmr} in this period. The background $\chem{O_3}$-\acrshort{vmr} in the Arctic winter (before polar sunrise) are typically around 30-40 ppb (\cite{Foster2001}), which can be seen from the observations at each of the stations. The BE-branch produces on average 0-10 ppb $\chem{O_3}$ at the different stations until the end of April. Seen in relation with Figures \ref{fig:vert_ALT}-\ref{fig:vert_ZEP}, which contain the vertical profiles ozone- and halogen species mixing ratios, it seems to be higher reactive halogen-activity in Period 1 than in Period 2. This is consistent with the low ozone-\acrshort{vmr}, but seen as February is the first month of Period 1, it is not consistent with the fact that \acrshort{be}-induced \acrshort{ode} need sunshine to occur. 

\medskip

Indications of the heterogeneous reactions (Reactions \ref{R:7} and \ref{R:8})\footnote{The reactions providing the basis of the autocatalytical cycle causing the bromine explosion (\cite{Simpson2015})}\footnote{These reactions were only active for $\chem{X} = \chem{Br}$ in this branch} can be seen in Period 1 for all stations. As \chem{HOBr} and $\chem{Br_2}$, the product of Reactions \ref{R:7} and \ref{R:8}, is seen mostly aloft, and only to a small extent at the ground level, it seems that the heterogeneous aerosol reaction (Reaction \ref{R:8}) is the most active. An attempt was made earlier to increase the efficiency of \ref{R:7} by decreasing the mixing layer height, $L_{mix}$ to 100 m and thereby increasing the deposition rate constant for \chem{HOBr}. Unfortunately, this does not seem to have worked. 

\medskip

Higher \chem{HBr}-\acrshort{vmr} is seen from around mid-March at all stations, which is when the other halogen-species generally disappear (with a few exceptions, that I will come back to). According to the box model experiments performed by \cite{CAO}, \chem{HBr} will be the dominating species left after a BE. The increased \chem{HBr}-\acrshort{vmr} is thus reasonable, also in magnitude compared to the findings of \cite{barrie}\footnote{\cite{barrie} found \chem{HBr} (filterable-\chem{Br} concentrations on the order of 10-100ng m$^{-3} \approx$ 10$^{-12}$ mol mol$^{-1}$}. The anti-correlation with ozone is also consistent with the findings by \cite{barrie}. 

\medskip

In Period 2 of 2001, the ozone \acrshort{vmr} increases at the ground level of the stations (Figure \ref{fig:2p_step5}). At Alert, Barrow and, to a lesser extent, Zeppelin, the ozone \acrshort{vmr} is comparable to observations. At Summit, the observed ozone \acrshort{vmr} is about twice as much as produced by the BE-branch. Interestingly, elevated values of \chem{BrCl} at Summit extends into Period 2, which is not seen at the other stations (in Figures \ref{fig:vert_ALT}-\ref{fig:vert_ZEP}). As the heterogeneous reactions with chlorine were disabled, this is not a product of Reaction \ref{R:8}. The \chem{BrCl} originates from aloft, and could therefore arise from the parameterized transport from the stratosphere.\footnote{The stratosphere was turned off for all the runs, but species originating from the stratosphere are set to climatological values a few kilometers above the tropopause (\cite{Sovde2012})} 

\medskip

During Period 2, the \chem{HBr}-\acrshort{vmr} remains at values of about 10-30 ppt at all stations before it seems to be transported from the column above the station (it could also have been oxidized by Reaction \ref{rqn:oh_hbr}, but as there is no sign of that in the \chem{Br} column, it seems unlikely). Following the decline in \chem{HBr}, there is an increase in $\chem{O_3}$ towards the end of June at all stations. 

\medskip

Ozonesonde measurements (in ppb) from Summit in 2013  were compared with the BE-branch and Original CTM3 model output run for the same year in Figure \ref{fig:ozonosonde2013}. The underestimation of $\chem{O_3}$\acrshort{vmr} demonstrated by the BE-branch in the runs for 2001 (Figure \ref{fig:2p_step5}) can also be seen in this figure. The large standard deviations seen in the lower layers indicate that the results across the Arctic are highly variable, as opposed to the Original CTM3 results. The large variations could in theory be due to \acrshort{ode} occuring at different locations in the Arctic, as this is an average over the whole area, in which case it would not be shown in the Original CTM3 results as it does not contain the halogen chemistry. As I don't have the full time-series for the BE-branch in 2013, this is very uncertain. However, even the errorbars do not reach the observed $\chem{O_3}$ mixing ratios whatsoever, which suggests that the BE-branch is depleting too much ozone either way.


\subsection{Analysis of the Difference Between the Final BE-Branch and the Original CTM3}\label{sec:disc_origBE}

The correlations between the model results (BE-branch and the original CTM3) for the Periods 1 and 2 shown in Figures \ref{fig:joint_FebApr_ALTSUM}-\ref{fig:joint_AprMay_ZEPBRW} generally depicts poor agreement between both the models and the observational data. There are no correlations above 0.5, and the correlation coefficient varies between being negative and positive. This motivates further development of the halogen chemistry causing \acrshort{ode} in the Arctic, as the halogen implementation in this thesis clearly does not capture them. Also, the correlations between the Original CTM3 and the observational data appears to be slightly higher (and more significant, according to significance if $p<0.05$) for most of the stations. 

\medskip

Figures \ref{fig:BE_origPD_vmrperc_FebApr}-\ref{fig:BE_origPD_vmrperc_AprJune} provide an overview of the monthly mean $\chem{O_3}$\acrshort{vmr} difference between the Original CTM3 and the BE-branch. The monthly mean ozone-\acrshort{vmr} produced by the BE-branch indicates that there is virtually no ozone in the lowest model-layer across the Arctic Ocean. This indicates that the ozone depletion is indeed "too effective". In Period 2 (Figure \ref{fig:BE_origPD_vmrperc_AprJune}), the BE-branch ozone mixing ratios regain some magnitude, and in May, the general picture is approximately 8-12 ppb. This is the same temporal tendency seen in Figure \ref{fig:2p_step5}. The difference between the Original CTM3 and the BE-branch is constantly positive however (except some outliers in February), and the Original CTM3 mixing ratios generally show expected background values (30-40 ppb) (e.g. \cite{Peterson2015}). 

\section{Radiative Forcing}\label{sec:disc_RF}

This section covers the estimated \acrshort{rf} due to tropospheric ozone for the BE-branch and the Original CTM3 in 2001 and 2013. As explained in Section \ref{sec:disc_step4}, the setup of the model for the \acrshort{pd} and \acrshort{pi} BE-branch differ due to model instabilities occurring when the \acrshort{pd} high Henry's law constant and higher photodissociation of \chem{HOBr} were applied to the \acrshort{pi} script and vice versa. Thus, the RF estimated from the BE-branch is not trustworthy, but is still conceptually analysed. When running the Original CTM3, the same setup was applied to both \acrshort{pi} and \acrshort{pd} runs. 

\medskip

The vertical profiles shown in Figure \ref{fig:vert_RF_2001} for 2001 and Figure \ref{fig:vert_RF_2013} for 2013 expresses the estimated ozone induced RF in each model layer averaged over the Arctic, Zeppelin, Barrow, Summit and Alert. The profiles show some dependence with altitude with regards to the magnitude of the RF. \cite{MYHRE2011387} found that \acrfull{bc}-induced RF (global annually averaged) with altitude would reach a maximum at approximately 800 hPa and be lower above and below (see Figure \ref{fig:Myhre2011} in Appendix \ref{app:supp_fig}). The shape and is seen in the averaged Arctic profile and at Alert in May and June with a maximum around 700 hPa. It is also seen at Barrow in March. This could indicate that there is a vertical dependence of ozone-induced RF. When the vertical profiles from 2013 are taken into account, however, it shows a completely different picture. The profile shape is an increasing RF with altitude, and there is little variation between the locations. 


\medskip

Differences between the BE-branch and the Original CTM3 in monthly averaged RF in the first model layer are shown for the year 2001 (Figures \ref{fig:BEorig_RF_polar_FebApr_2001}-\ref{fig:BEorig_RF_polar_AprJune_2001}) and 2013 (Figures \ref{fig:BEorig_RF_polar_FebApr_2013}-\ref{fig:BEorig_RF_polar_AprJune_2013}). In 2001, the Original CTM3 produces higher RF than the BE-branch, which was expected, as the ozone \acrshort{vmr} at the ground level in the Arctic seen in Figures \ref{fig:BE_origPD_vmrperc_FebApr}-\ref{fig:BE_origPD_vmrperc_FebApr} consistently showed higher values produced by the Original CTM3. It is also in agreement with the effects \cite{Sherwen2017} found after implementing halogen chemistry, which was a reduction in the ozone-induced RF. In 2013, however, the BE-branch produces ozone-induced RF-values comparable to the Original CTM3, and the difference between the two is relatively small. 

\medskip


The averaged RF globally and over the Arctic, produced by the BE-branch and the Original CTM and the difference between them for the whole period, Period 1 and Period 2 for 2001 and 2013 is shown in Table \ref{tab:RF_results}. In 2001, large variations are exhibited whether the BE-branch considers the whole globe or just the Arctic. Globally the BE-branch produces negative RF, which is occurring from Period 2 (Period 1 is positive). When the Arctic is considered, the BE-branch produces positive RF, but this time with more heating in Period 2. Averaged over the Arctic and globally, the Original CTM3 produces much higher RF-values than the BE-branch. In 2013, the BE-branch produces higher RF-values than the Original CTM3 globally and over the Arctic for all the periods except in Period 1 over the Arctic. The BE-branch generally have standard deviations larger than, or comparable in size to the estimated RF. 

\medskip

The temporally (February to June) and globally averaged RF due to tropospheric ozone increases estimated by the BE-branch is $-0.012\pm0.12$ Wm$^{-2}$ in 2001 and $0.45\pm0.42$ Wm$^{-2}$ in 2013. Averaged over the Arctic and the whole time period, RF due to tropospheric ozone increases estimated by the BE-branch is $0.065\pm0.069$ Wm$^{-2}$ in 2001 and $0.55\pm0.70$ Wm$^{-2}$ in 2013.

The temporally (February to June) and globally averaged RF due to tropospheric ozone yielded by the BE-branch demonstrates large deviations between the 2001-run and the 2013-run. The 2001 RF of $-0.012\pm0.12$ Wm$^{-2}$ indicates global ozone-induced tropospheric cooling, which is not reasonable. The estimate from 2013 of $0.45\pm0.42$ Wm$^{-2}$ is closer to what what was reported by the IPCC (\cite{IPCCchapter8}) of $0.40\pm0.20$ Wm$^{-2}$ (global annual average). However, seen as the discrepancy between the two runs are so large, neither of the estimates are trustworthy. The inconsistency is only made clearer by looking at the global RF for each of the runs  (Figure \ref{fig:BE_RF_global_2001} for 2001 and \ref{fig:BE_RF_global_2013} in Appendix \ref{app:RF}). The 2001-run demonstrates global average cooling, with slight warming in the Arctic, whereas the 2013-run exhibits warming practically everywhere, reaching unrealistic high ozone induced RF-values in June. The reason for this inconsistency between the two runs is uncertain. The runs are started from the same restart-file, and seen as the Original CTM3 does not produce the same discrepancy, it is likely that the instability demonstrated by the BE-branch when making production runs (see Section \ref{sec:disc_step4}). This instability was also the reason behind the different setups of the \acrshort{pd}- and \acrshort{pi}-runs, which is also a reason why these RF-estimates are unreliable.


% Furthermore, \cite{Sherwen2017} found that including halogen chemistry yielded an annual global averaged RF of 0.345  Wm$^{-2}$. Thus, a lower RF-estimate is not unreasonable. Due to the large deviation between 2001 and 2013 produced by the same branch, however, the estimate is rather untrusworthy. The Original CTM3 yields 0.10 Wm$^{-2}$ lower than the IPCC estimate.\footnote{Note that the estimated RF from the BE-branch is only averaged over 5 months} \cite{shindell2003} reported the Arctic annual mean RF to be $0.24-0.4$ Wm$^{-2}$, which does not agree with the 2001 BE-branch RF over the Arctic ($0.039\pm0.046$ Wm$^{-2}$). The 2013 BE-branch RF over the Arctic ($0.24\pm0.18$ Wm$^{-2}$) agrees more with this estimate, but again, the large deviation between the two makes the BE-branch calculation unreliable.  


\section{Future work}

The suggestions for future work are divided into three parts. First, implementing physical processes important for the halogen-induced \acrfull{ode}, which were not taken into account deliberately, as it would have been too extensive for the scope of this thesis. Second, specifying processes that were implemented, but simplified to a large extent. Lastly, there are some suggestions for analysis of the final BE-branch, and how the results could have been improved.

\subsection{Physical Processes That Could Have Been Implemented}

Some specifics considering the nature of the \acrshort{ode} and \acrfull{be}s were not taken into account, but are advisable to look into for future work. This includes: 
\begin{itemize}
    \item The acidity of the reaction surface for the heterogeneous reactions were not considered, although the efficiency of these reactions are highly dependent on the pH of the reaction surface (\cite{KerriAPratt2013}).
    \item \cite{Parella} suggested sea salt debromination increase from \acrlong{pi} times to \acrlong{pd} due to enhanced particle acidity from \acrlong{pd} emissions. This was not considered in this implementation.
    \item Antropogenic emissions of organic halocarbons as a source for reactive bromine was not taken into account, only the organic halocarbons originating from the ocean (explained in \ref{sec:oceanic_emissions}).  
\end{itemize}

\subsection{Physical Processes That Were Implemented But Simplified}

Some processes were deliberately simplified to a large extent when implemented in the CTM3. Future work is advised to take into consideration the following: 

\begin{itemize}
    \item The implementation in Section \ref{sec:oceanic_emissions} does not take into consideration antropogenic emissions of organic halocarbons, seasonality in emissions or the difference in lifetime for $\chem{CHBr_3}$ and $\chem{CH_2Br_2}$. The implementation is a latitudinal fixed parameterization, and thus cannot interchange with existing inventories in the model. It would have been better to use an NetCDF-based emission field. 
    \item In the implementation of the heterogeneous reactions over snow and ice (Section \ref{sec:tropchem_oslo}), the existence of sea ice is the only variable determining the occurrence of the reaction. To improve this, the age of the ice (multiyear ice or newly formed ice), snow cover on the ice and pH of the ice should be considered (e.g. \cite{Thomas2011}, \cite{Peterson2019}).
    \item The parameterization of the aerodynamic resistance $r_a$ in Sections \ref{sec:snow_ice_react} should have been calculated using values for the boundary layer conditions, wind speed and boundary layer height, that are in the CTM3 already, rather than prescribed values from \cite{CAO}. 
    \item In order to calculate the deposition rate constant of \chem{HOBr} onto snow-and ice surfaces, the parameterization of this constant when implemented was the same as used by the box-model experiment by \cite{CAO}. Thus, the deposition rate was constant according to what I used as $\beta$, $L_{mix}$ and $v_d$ (see Section \ref{sec:snow_ice_react}). The mixing layer height, $L_{mix}$, and thereby the deposition velocity, $v_d$, should at least be possible to calculate from the CTM3. 
\end{itemize}


\subsection{The Final BE-Branch}


The model should have been run for a full year rather than six months to obtain a better idea of the seasonal variability. To have the best possible comparison basis, the same restart file had to be used for all the branches.\footnote{Used in a study by \cite{Falk_2019}. The setup is explained in \ref{subsec:restart_files}} Ideally, the runs should have had a spin-up period of approximately a model month before being used. As this was not done, the January data were regarded as a spin-up rather than being used in further analysis (See Figure \ref{fig:ozone_2001_step4}. The BE-branch is clearly affected by the lack of spin-up). Had the model been run for a full year, it would have been possible to include Antarctic observations (the Antarctic also experiences ODEs (\cite{Simpson2015})).

\medskip
I was not able to do a run to include the observational data from Eureka. Ideally the whole 6-month period should have been included for the year 2013 to be able to get a robust picture of the comparison of model data against the observations. Due to lack of time, I chose to prioritize the RF-calculations and comparison with ozonesone-data for this year. As there are ozone observations available for this station, I suggest these could have been used for comparison in future work. 

\medskip

The halogen-chemistry implementation was clearly not working perfectly, as the model experienced negative values of the \texttt{ISOR1}-radical by only slight changes in the wet deposition and photodissosiation explained in Sections \ref{sec:res_step4} and \ref{sec:res_final_Version}. Due to this, the \acrshort{pd} and \acrshort{pi} BE-branch were ran with different setups, which again affects the RF-calculation, providing unreliable results. Future work would be advised to find a stable solution that works for both setups. Moreover, the final version of the BE-branch did not include the heterogeneous Cl-chemistry, which it should have, as chlorine species most likely has an impact on the halogen-induced tropospheric ozone depletions (e.g. \cite{FinlaysonPitts2010}).