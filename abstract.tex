\chapter*{Abstract}
\addcontentsline{toc}{chapter}{Abstract}


%\begin{itemize}
%    \item Introduce topic and why its important
%    \item Introduce a challenge or unresolved issue that you will try and solve
%    \item What have you done to try and solving this
%    \item Main result - Include the numerical result of your best model
%    \item The implications in the context of 1+2
%\end{itemize}

Increased anthropogenic emissions of nitrogen oxides, carbon monoxide, volatile organic compounds and methane causes production of ozone in the troposphere, particularly in the northern hemisphere. Tropospheric ozone acts as a greenhouse gas. Ozone depletion events (ODEs) caused by reactive halogens in the Arctic are a well known and thoroughly studied phenomena, which may modulate the warming effect of tropospheric ozone in the Arctic. The purpose of this thesis is to implement the halogen chemistry causing the ODEs in the Arctic, as they are currently lacking in the Oslo CTM3 (chemical transport model) and thereby attempt to reproduce the ODEs by comparing with observations. If the implementation produces reasonable results, the model can be used to further analyse the effect these events may have on the radiative balance in the Arctic. 
\cleardoublepage


% \begin{itemize}
%     \item Develop a reliable halogen-chemistry scheme in the Oslo CTM3 and estimate performance of the new scheme by comparing with observations and the original CTM3
%     \item Comparing the runs with pre-industrial conditions, as tropospheric ozone mainly occurs due to anthropogenic emissions of $\chem{NO_x}$, \chem{VOC}s, \chem{CO} and $\chem{CH_4}$.
%     \item Calculate the ozone-induced RF field to find out whether the ODE-implementation causes changes in the ozone induced RF 
%     \item Compare pre-industrial (pre-industrial is here defined as pre-1850) and present day ozone induced RF with the original and modified CTM3
% \end{itemize}