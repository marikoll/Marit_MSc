\chapter*{Abstract}
\addcontentsline{toc}{chapter}{Abstract}


%\begin{itemize}
%    \item Introduce topic and why its important
%    \item Introduce a challenge or unresolved issue that you will try and solve
%    \item What have you done to try and solving this
%    \item Main result - Include the numerical result of your best model
%    \item The implications in the context of 1+2
%\end{itemize}

Increased anthropogenic emissions of nitrogen oxides, carbon monoxide, volatile organic compounds and methane causes production of ozone in the troposphere, particularly in the northern hemisphere. This causes warming of the northern hemisphere troposphere as tropospheric ozone acts as a greenhouse gas. Ozone depletion events (ODEs) caused by reactive halogens in the Arctic are a well known and thoroughly studied phenomena, which may modulate the warming effect of tropospheric ozone in the Arctic. The purpose of this thesis is to develop a reliable halogen-chemistry scheme in the Oslo CTM3 and estimate performance of the new scheme by comparing with observations and the original CTM3. The ozone-induced radiative forcing (RF) due to the implemented halogen chemistry can then be estimated. The new scheme was run for 2001 and 2013, but only the 2001 run was compared with observations. The new scheme shows no significant correlation with ground-based measurements of ozone. In the new scheme, \chem{HBr} appears to be the dominant halogen species during and after ODEs. Compared to the Original CTM3, the new scheme produces about 20-30 ppb less $\chem{O_3}$. Diverging results in the 2001 and 2013 run, makes i specific dependence with altitude regarding tropospheric ozone-induced RF. The temporally (February to June) and globally averaged RF due to tropospheric ozone yielded by the BE-branch demonstrates large deviations between the 2001-run, RF $=-0.11\pm0.11$ Wm$^{-2}$, and the 2013-run, RF $= 0.39\pm0.31$ Wm$^{-2}$. Due to the inconsistency in RF and the fact that the present-day and pre-industrial setup of the BE-branch, these estimates are not correct. The new scheme does not sufficiently reproduce the observed ODEs in the Arctic, and the RF-estimates are variable and inconsistent. 
\cleardoublepage

